%%%========================== %%%
%%%  Set Beamer as class        
%%%==========================%%%
%%
%% Beamer
\documentclass{beamer}
\setbeamercovered{transparent}

%%%========================== %%%
%%%  Set theme        
%%%==========================%%%
%%
%% Warsaw
\usetheme{Warsaw}

%%%========================== %%%
%%%  German Umlauts                            
%%%==========================%%%
%%
%% For Mac OS X
%% \usepackage[applemac]{inputenc}
%% For PC Windows
%% \usepackage[ansinew]{inputenc}
%%
%% For PC Linux 
%% \usepackage[latin1]{inputenc}
%%
%% For UTF-8
\usepackage[utf8]{inputenc}

%%%========================== %%%
%%%  German hyphenation              
%%%==========================%%%
%%
\usepackage{ngerman}

%%%========================== %%%
%%%  Quotes
%%%  \enquote{}       
%%%==========================%%%
%%
\usepackage[babel,german=quotes]{csquotes}

%%%========================== %%%
%%%  Tikz   
%%%==========================%%%
%%
\usepackage{xcolor}
\usepackage{tikz,pgffor}
\usetikzlibrary{shadows}
\tikzset{
	MyPersp/.style={scale=1.8,x={(-0.8cm,-0.4cm)},y={(0.8cm,-0.4cm)},
    z={(0cm,1cm)}},
%  MyPersp/.style={scale=1.5,x={(0cm,0cm)},y={(1cm,0cm)},
%    z={(0cm,1cm)}}, % uncomment the two lines to get a lateral view
	MyPoints/.style={fill=white,draw=black,thick}
		}

% #1 number of teeths
% #2 radius blanket intern
% #3 radius blanket extern
% #4 angle from start to end of the first arc
% #5 angle to decale the second arc from the first 
% #6 radius dna intern
% #7 radius dna extern
% #8 xshift
\newcommand{\gear}[8]{%
  \foreach \i in {1,...,#1} {%
    [rotate=(\i-1)*360/#1]  (0:#2)  arc (0:#4:#2) {[rounded corners=2pt]
     -- (#4+#5:#3)  arc (#4+#5:360/#1-#5:#3)} --  (360/#1:#2)
  }%
  (0,0) circle[radius=#6]
  (0,0) circle[radius=#7];
\draw[thick,xshift=#8]
\foreach \i in {1,2,...,36} {%
  [rotate=(\i-1)*10]  (0.0,#6) -- (0.0,#7)
};
}  

\usepackage{hyperref}


%%%========================== %%%
%%%  Settings for the startpage              
%%%==========================%%%
%%
\title{Introducing: \enquote{The Machine} (DRAFT)}
\author{\texorpdfstring{\ Jürgen Scholz \ \newline\url{j.scholz@machinecoin.org}}{Author}}
\institute{The Machine - A complete self-contained cryptographic ecosphere driven by nothing but the Machinecoin cryptocurrency.}
\date{2014/06/02}

%%%========================== %%%
%%%  Document Start                                
%%%==========================%%%
%%
\begin{document}
\frame{\titlepage}

\frame
{
\tableofcontents
}

\section{Introduction}

\frame
{   
In November 2008, a paper titled ``Bitcoin: A Peer-to-Peer Electronic Cash System'' was posted on The Cryptography Mailing List at metzdowd.com. It was written by either a person or group with the pseudonymous ``Satoshi Nakamoto''. In this paper methods of using a peer-to-peer network to generate a system for electronic transactions without relying on trust were described in detail.
\newline
\newline
\visible<2->{\texttt{``A purely peer-to-peer version of electronic cash would allow online payments to be sent directly from one party to another without going through a financial institution.''}
  \vskip5mm
  \hspace*\fill{\small--- Satoshi Nakamoto, Bitcoin: A Peer-to-Peer Electronic Cash System\footnote{http://bitcoin.org/bitcoin.pdf}}}
}


%\section{Bibliography}
%\begin{frame}
%	\frametitle{Quellen}
%	\bibliographystyle{alpha}
%	\bibliography{literatur}		
%\end{frame} 

\end{document}
%%%%%%%%%%%%%%%%%%%%%%
%%%  Dokument Ende                           
%%%%%%%%%%%%%%%%%%%%%%